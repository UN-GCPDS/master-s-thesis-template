\subsubsection{USB-Serial dongle}

\textit{OpenBCI Cython} defaults interface is serial, then the endpoint is a serial port.
\begin{python}
from openbci_stream.acquisition import Cyton
openbci = Cyton('serial', endpoint='/dev/ttyUSB0', capture_stream=True)
\end{python}

%======================================================================
\subsubsection{Wi-Fi shield}

The Wi-Fi shield supports two modes: slave and access point, whatever mode the endpoint is an IP address.
\begin{python}
from openbci_stream.acquisition import Cyton
openbci = Cyton('wifi', endpoint='192.68.1.113', capture_stream=True)
\end{python}

%======================================================================
\subsubsection{Synchronous mode}

The synchronous mode it implies blocking call, the connection must be closed with \quot{openbci.close()}.
\begin{python}
from openbci_stream.acquisition import Cyton

openbci = Cyton('wifi', endpoint='192.68.1.113', capture_stream=True)
openbci.stream(15)
\end{python}

%======================================================================
\subsubsection{Asynchronous mode}

Under asynchronous mode the acquisition runs in background (using subprocess), then, others tasks can be performed in parallel.
\begin{python}
from openbci_stream.acquisition import Cyton

openbci = Cyton('wifi', endpoint='192.68.1.113', capture_stream=True)
openbci.start_stream()
time.sleep(15)  # collect data for 15 seconds
openbci.stop_stream()
\end{python}

%======================================================================
\subsubsection{Test mode}

The test mode generates synthetic square waveforms.
\begin{python}
from openbci_stream.acquisition import Cyton
from openbci_stream.acquisition import CytonConstants as CONST

openbci = Cyton('wifi', endpoint='192.68.1.113', capture_stream=True)

openbci.command(CONST.TEST_1X_FAST)
# openbci.command(CONST.TEST_1X_SLOW)
# openbci.command(CONST.TEST_2X_FAST)
# openbci.command(CONST.TEST_2X_SLOW)
\end{python}

%======================================================================
\subsubsection{Set sample frequency and transmission package size}

The sample frequency for the Wi-Fi interface can be configured up to 16 kSPS. In this example the sample frequency was configured to acquire at 1 KSPS and the streaming package size at 250, this means that a package of 250 samples will be streamed each 250 ms.
\begin{python}
from openbci_stream.acquisition import Cyton
from openbci_stream.acquisition import CytonConstants as CONST

openbci = Cyton(
    'wifi',
    endpoint='192.68.1.113',
    capture_stream=True,
    streaming_package_size=250,
)

# openbci.command(CONST.SAMPLE_RATE_250SPS)
# openbci.command(CONST.SAMPLE_RATE_500SPS)
openbci.command(CONST.SAMPLE_RATE_1KSPS)
# openbci.command(CONST.SAMPLE_RATE_2KSPS)

\end{python}


% %======================================================================
% \subsubsection{Context managers}

% \begin{python}
% from openbci_stream.acquisition import OpenBCIConsumer
% import time

% with OpenBCIConsumer(
%     mode='serial',
%     endpoint='/dev/ttyUSB0',
%     streaming_package_size=250) as (stream, openbci):
%     t0 = time.time()
    
%     for i, message in enumerate(stream):
%         if message.topic == 'eeg':
%             print(f"{i}: received {message.value['samples']} samples")
%             t0 = time.time()
%             if i == 9:
%                 break
% \end{python}

% %======================================================================
% \subsubsection{Write data}

% \begin{python}

% \end{python}

% %======================================================================
% \subsubsection{Read data}

% \begin{python}

% \end{python}

%======================================================================
\subsubsection{Impedance measurement}


\begin{python}
from openbci_stream.acquisition import Cyton
from openbci_stream.acquisition import CytonConstants as CONST
import numpy as np
import time

def get_z(v):
    rms = np.std(v)
    z = (1e-6 * rms * np.sqrt(2) / 6e-9) - 2200
    if z < 0:
        return 0
    return z

openbci = Cyton('wifi', endpoint='192.68.1.113', capture_stream=True)

# configure the lead-off impedance for the ADS1299
openbci.command(CONST.SAMPLE_RATE_250SPS)
openbci.command(CONST.DEFAULT_CHANNELS_SETTINGS)
openbci.leadoff_impedance(
    range(1, 9),
    pchan=CONST.TEST_SIGNAL_NOT_APPLIED,
    nchan=CONST.TEST_SIGNAL_APPLIED,
)

# non-blocking call
openbci.start_stream()  
while True:
    time.sleep(1)  # collect data for 1 seconds
    #  Calculate the impedance for the last second
    z = get_z(openbci.eeg_time_series[-1000:][0])  # first channel
    print(f'{z/1000:.2f} kOhm')
\end{python}