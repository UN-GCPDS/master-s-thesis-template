% Preview source code for paragraph 6

\begin{table}
\begin{centering}
\begin{tabular}{>{\raggedleft}m{1.3cm}>{\centering}m{1cm}>{\centering}m{1cm}>{\centering}m{1cm}>{\centering}m{1cm}>{\centering}m{1cm}>{\centering}m{1cm}>{\raggedright}m{3.8cm}}
\toprule 
\textbf{Footer byte {[}33{]}} & \textbf{Byte 26} & \textbf{Byte 27} & \textbf{Byte 28} & \textbf{Byte 29} & \textbf{Byte 30} & \textbf{Byte 31} & \textbf{Description}\tabularnewline\addlinespace[1em]
\midrule
\addlinespace[0.5cm]
\quottable{0xc0} & AX1 & AX0 & AY1 & AY0 & AZ1 & AZ0 & Standar with accel\tabularnewline\addlinespace[1em]
\addlinespace[0.5cm]
\quottable{0xc1} & UDF & UDF & UDF & UDF & UDF & UDF & Standar with raw aux\tabularnewline\addlinespace[1em]
\addlinespace[0.5cm]
\quottable{0xc2} & UDF & UDF & UDF & UDF & UDF & UDF & User defined\tabularnewline\addlinespace[1em]
\addlinespace[0.5cm]
\quottable{0xc3} & AC & AV & T3 & T2 & T1 & T0 & Timestamp \emph{set} with accel\tabularnewline\addlinespace[1em]
\addlinespace[0.5cm]
\quottable{0xc4} & AC & AV & T3 & T2 & T1 & T0 & Timestamp with accel\tabularnewline\addlinespace[1em]
\addlinespace[0.5cm]
\quottable{0xc5} & UDF & UDF & T3 & T2 & T1 & T0 & Timestamp \emph{set} with raw aux\tabularnewline\addlinespace[1em]
\addlinespace[0.5cm]
\quottable{0xc6} & UDF & UDF & T3 & T2 & T1 & T0 & Timestamp with raw aux\tabularnewline\addlinespace[1em]
\bottomrule
\addlinespace[0.5cm]
\end{tabular}
\par\end{centering}
\caption{The type of auxiliar data based on the 33-byte of the binary pakage.\label{table:auxiliar_data}}
\end{table}


