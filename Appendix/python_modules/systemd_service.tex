%======================================================================
\chapter{Python: Systemd service}\label{appendix:python_systemd_service}

\begin{description}
   \item[Description:]      Simple API to automate the creation of custom daemons for GNU/Linux.
   \item[License:]          BSD-2-clause
   \item[Latest version:]   \quot{1.8}
   \item[Python:]           \quot{3.8, 3.9, 3.10}
   \item[PyPi:]             \url{https://pypi.org/project/systemd-service/}
   \item[Repository:]       \url{https://github.com/UN-GCPDS/systemd-service}
   \item[Documentation:]    \url{https://systemd-service.readthedocs.io/en/latest/}
\end{description}
\hrulefill

A daemon is a service process that runs in the background and supervises the system or provides functionality to other processes. Traditionally, daemons are implemented following a scheme originating in SysV Unix \cite{daemon73:online}. Modern daemons should follow a simpler yet more powerful scheme, as implemented by systemd \cite{systemd78:online}.

\textit{Systemd service} is a Python module to automate the creation of Python-based daemons under GNU/Linux environments.


%======================================================================
\section{Install}
\begin{python}
pip install -U systemd-service
\end{python}


%======================================================================
\section{Handle daemons}
\begin{python}
from systemd_service import Service

daemon = Service("stream_rpyc")

daemon.stop()     # Start (activate) the unit.
daemon.start()    # Stop (deactivate) the unit.
daemon.reload()   # Reload the unit.  
daemon.restart()  # Start or restart the unit.

daemon.enable()   # Enable the unit.
daemon.disable()  # Disable the unit.

daemon.remove()   # Remove the file unit.
\end{python}
This commands are uquivalent to the \quot{systemctl} calls, for example run in terminal the folowing command:
\begin{python}
\$ systemctl enable stream_rpyc
\end{python}
Can be running inside a Python environment with using \quot{systemd\_service}
\begin{python}
from systemd_service import Service

daemon = Service("stream_rpyc")
daemon.enable()
\end{python}


%======================================================================
\section{Creating services}
Similar to the previous scripts, the services can be created using \quot{systemd\_service}:

\begin{python}
daemon = Service("stream_rpyc")
daemon.create_service()
\end{python}
If the service must be initialized after other service
\begin{python}
daemon = Service("stream_rpyc")
daemon.create_service(after='ntpd')
\end{python}


%======================================================================
\section{Creating timers}
Defines a timer relative to when the machine was booted up:
\begin{python}
daemon = Service("stream_rpyc")
daemon.create_timer(on_boot_sec=15)
\end{python}


%======================================================================
\section{Example}
This module is useful when is combined with package scripts declaration in \quot{setup.py} file:
\begin{python}
# setup.py

scripts=[
    "cmd/stream_rpyc",
]
\end{python}
The script could looks like:
\begin{python}
#!/usr/bin/env python

import sys

if sys.argv[-1] == "systemd":
    from systemd_service import Service
    daemon = Service("stream_rpyc")
    daemon.create_timer(on_boot_sec=10, after='network.target kafka.service')

else:
    from my_module.submodule import my_service
    print("Run 'stream_rpyc systemd' as superuser to create the daemon.")
    my_service()
\end{python}
Then the command can be called as a simple script but with the \quot{systemd} argument the command will turn into a service.
\begin{python}
\$ stream_rpyc
# Command executed normally
\end{python}
\begin{python}
\$ stream_rpyc systemd
# Service created
\end{python}