%======================================================================
\chapter{Python: Matplotlib-FigureStream}\label{appendix:matplotlib-figurestream}

\begin{description}
   \item[Description:]      A backend for serve Matplotlib animations as web streams.
   \item[License:]          BSD-2-clause
   \item[Latest version:]   \quot{1.2.6}
   \item[Python:]           \quot{3.8, 3.9, 3.10}
   \item[PyPi:]             \url{https://pypi.org/project/figurestream/}
   \item[Repository:]       \url{https://github.com/UN-GCPDS/matplotlib-figurestream}
   \item[Documentation:]    \url{https://figurestream.readthedocs.io/en/latest/}
\end{description}
\hrulefill

%======================================================================
\section{Install}
\begin{python}
pip install -U figurestream
\end{python}

%======================================================================
\section{Bare minimum}
By default, the stream serves on \url{http://localhost:5000}

\begin{python}
# FigureStream replace any Figure object
from figurestream import FigureStream

import numpy as np
from datetime import datetime

# FigureStream can be used like any Figure object
stream = FigureStream()
sub = stream.add_subplot(111)
x = np.linspace(0, 3, 1000)

# Update animation loop
while True:
    sub.clear()  # clear the canvas

    # ------------------------------------------------------------------------
    # Any plot operation
    sub.set_title('FigureStream')
    sub.set_xlabel('Time [s]')
    sub.set_ylabel('Amplitude')
    sub.plot(x, np.sin(2 * np.pi * 2 * (x + datetime.now().timestamp())))
    sub.plot(x, np.sin(2 * np.pi * 0.5 * (x + datetime.now().timestamp())))
    # ------------------------------------------------------------------------

    stream.feed()  # push the frame into the server
\end{python}

For fast updates is recommended to use \quot{set\_data}, \quot{set\_ydata} and \quot{set\_xdata} instead of clear and draw again in each loop, also \quot{FigureStream} can be implemented from a custom class.

\begin{python}
# FigureStream replace any Figure object
from figurestream import FigureStream

import numpy as np
from datetime import datetime


class FastAnimation(FigureStream):
    def __init__(self, *args, **kwargs):
        super().__init__(*args, **kwargs)

        axis = self.add_subplot(111)
        self.x = np.linspace(0, 3, 1000)

        # ---------------------------------------------------------------------
        # Single time plot configuration
        axis.set_title('FigureStream')
        axis.set_xlabel('Time [s]')
        axis.set_ylabel('Amplitude')

        axis.set_ylim(-1.2, 1.2)
        axis.set_xlim(0, 3)

        # Lines objects
        self.line1, *_ = axis.plot(self.x, np.zeros(self.x.size))
        self.line2, *_ = axis.plot(self.x, np.zeros(self.x.size))
        # ---------------------------------------------------------------------

        self.anim()

    def anim(self):
        # Update animation loop
        while True:
            # -----------------------------------------------------------------
            # Update only the data values is faster than update all the plot
            self.line1.set_ydata(
                np.sin(2 * np.pi * 2 * (self.x + datetime.now().timestamp()))
            )
            self.line2.set_ydata(
                np.sin(
                    2 * np.pi * 0.5 * (self.x + datetime.now().timestamp())
                )
            )
            # -----------------------------------------------------------------

            self.feed()  # push the frame into the server


if __name__ == '__main__':
    FastAnimation()

\end{python}

%======================================================================
\section{Set host, port and endpoint}
If we want to serve the stream in a different place we can use the parameters \quot{host}, \quot{port} and \quot{endpoint}, for example:
\begin{python}
FigureStream(host='0.0.0.0', port='5500', endpoint='figure.jpeg')
\end{python}

Now the stream will serve on \url{http://localhost:5500/figure.jpeg} and due the \quot{0.0.0.0} host is accesible for any device on network.
By default \quot{host} is \quot{localhost}, \quot{port} is \quot{5000} and \quot{endpoint} is empty.
