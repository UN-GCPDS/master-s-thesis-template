\cleardoublepage
\chapter*{Resumen}
\markboth{Resumen}{}
\addcontentsline{toc}{chapter}{Resumen}

% The widespread use of neurophysiological signals to develop \gls*{BCI} systems has certainly varied clinical and nonclinical applications. Main implementations in medical issues include: rehabilitation, cognitive state analysis, diagnostics, assistive devices for communication, locomotion and movement. By other hand, there is a bunch of researches that approaches the \gls*{BCI} systems to healthy people in fields like: neuroergonomics, smart homes, neuromarketing and advertising, games, education, entertainment and even security and validation. Not all EEG acquisition systems are capable to use in \gls*{BCI}s systems. Even if the clinic devices are highly accurate, these implementations have a limited, or nonexistent, real-time data flow access; because they mainly use is about diagnostic and offline analysis. Recently, and because of the cheapening prototyping development, there is in the market a set of low-cost embedded systems for \gls*{EEG} acquisition, i.e., OpenBCI, InteraXon, Muse, NeuroSky MindWave and Emotiv. All these options usually include a high or low-level \gls*{SDK}, that could be open-source or proprietary and will come with a different grade of flexibility (rigid or customizable electrode placement, multiple sampling rates, transmission protocols, wireless, etc). Many of these devices have shown capabilities to handle \gls*{BCI} tasks, but they need a context-specific development to boost their base benefits. Acquiring brain signals is only one task for a \gls*{BCI} system, also it is necessary to carry out a lot of data processing and controlled experiments, concerning this have been specialized software for developers and researchers purpose i.e., BCI2000, Neurobehavioral Systems Presentation, Psychology Software Tools, Inc. ePrime and PsychoPy. All these systems offer greater ease of use through experimenter interfaces, but they can be costly, require high-level programming and technical skills, and usually do not support dedicated data acquisition. For this reason, the acquisition involves the implementation of third party software and drivers; consequently, losing interesting hardware features in favor to support as many devices as possible. To implement a \gls*{BCI} system is an interdisciplinary activity that requires a set of specific and outstanding knowledges about communication systems, signals acquisition, instrumentation, clinical protocols, experiments validation, software development, among others.\\

El uso generalizado de señales neurofisiológicas para desarrollar sistemas \gls*{BCI} ciertamente tiene diversas aplicaciones clínicas y no clínicas. Las principales implementaciones en temas médicos incluyen: rehabilitación, análisis del estado cognitivo, diagnóstico, dispositivos de asistencia para la comunicación, locomoción y movimiento. Por otro lado, hay muchas investigaciones que acercan los sistemas \gls*{BCI} a personas sanas en campos como: neuroergonomía, hogares inteligentes, neuromarketing y publicidad, juegos, educación, entretenimiento e incluso seguridad y validación. No todos los sistemas de adquisición de EEG se pueden usar en los sistemas \gls*{BCI}s. Incluso si los dispositivos clínicos son muy precisos, estas implementaciones tienen un acceso limitado o inexistente al flujo de datos en tiempo real; debido principalmente a que se tratan sistemas enfocados al diagnóstico y análisis fuera de línea. Recientemente, y debido al abaratamiento del desarrollo de prototipos, existe en el mercado un conjunto de sistemas embebidos de bajo costo para la adquisición de \gls*{EEG}, algunos de ellos son: OpenBCI, InteraXon, Muse, NeuroSky MindWave y Emotiv. Todas estas opciones suelen incluir un \gls*{SDK} de nivel alto o bajo, que puede ser de código abierto o privativo los cuales vienen con un grado diferente de flexibilidad (disposición de electrodos rígida o personalizable, frecuencias de muestreo variable, diferentes protocolos de transmisión, conexión inalámbrica, etc). Muchos de estos dispositivos han demostrado capacidades para manejar tareas \gls*{BCI}, pero necesitan un desarrollo específico del contexto para aumentar sus beneficios básicos. Adquirir señales cerebrales es sólo una tarea indiidual para un sistema completo de \gls*{BCI}, también es necesario llevar a cabo una gran cantidad de procesamiento de datos y experimentos controlados, con respecto a esto se ha especializado software para desarrolladores e investigadores, por ejemplo: BCI2000, Neurobehavioral Systems Presentación, Psychology Software Tools, Inc. ePrime y PsychoPy. Todos estos sistemas ofrecen una mayor facilidad de uso a través de las interfaces del sistema de experimentos, pero pueden ser costosos, requieren habilidades técnicas y de programación de alto nivel y por lo general, no admiten la adquisición de datos dedicada. Por esta razón, la adquisición de señales se basa en la implementación de software y controladores de terceros; en consecuencia, se pierden características de hardware interesantes a favor de soportar tantos dispositivos como sea posible. Implementar un sistema \gls*{BCI} es una actividad interdisciplinaria que requiere un conjunto de conocimientos específicos y sobresalientes sobre sistemas de comunicación, adquisición de señales, instrumentación, protocolos clínicos, validación de experimentos, desarrollo de software, entre otros.\\


% Besides, in order to perform a real-world experiment, the user must calibrate the specific set of acquisition system, stimuli delivery and data processing stages. Current software approaches try to converge multiple technologies and methodologies to provide general purpose \gls*{BCI} systems. The most popular is the BCI200, which comes with default paradigms but their interface has been pointed out to be not very intuitive and its operation is difficult to understand, although, it is possible to add new paradigms, this include software contributions using their own libraries and do not thought a built-int development interface. Other software widely used is the OpenVIBE this one includes a graphical drag-and-drop interface to perform data analysis with an extensive set of pre-defined algorithms. Its synchronous acquisition system is known for not only occasionally frozen the computer but also for adding delays to the streaming of the signals. All these systems handle with an extensive set of compatible devices which may be good at first glance but make that some specific hardware features are not available for compatibility reasons. On the side of the open source hardware, we can find that OpenBCI a flexible option, but with some important lacks. The most important relies on the communication between the computer and the board is not always stable and their \gls*{GUI} does not provide the possibility of acquiring data under wich a particular \gls*{BCI} paradigm. Otherwise, their hardware base and \gls*{SDK} features gives to this board a huge potential to implement a complete \gls*{BCI} system comparable with medical grade equipment.\\

Además, para realizar un experimento del mundo real, el usuario debe calibrar el conjunto específico de sistema de adquisición, entrega de estímulos y etapas de procesamiento de datos. Los enfoques de software actuales intentan hacer converger múltiples tecnologías y metodologías para proporcionar sistemas \gls*{BCI} de propósito general. El más popular es el BCI200, que incorpora paradigmas predeterminados pero se ha señalado que su interfaz es poco intuitiva y su funcionamiento es difícil de entender, aunque es posible agregar nuevos paradigmas, esto permite incluir contribuciones de software utilizando sus propias bibliotecas. y no mediante una interfaz de desarrollo integrada. Otro software ampliamente utilizado es OpenVIBE, este incluye una interfaz gráfica de arrastrar y soltar para realizar análisis de datos con un amplio conjunto de algoritmos predefinidos. Su sistema de adquisición sincrónica es conocido no sólo por congelar ocasionalmente la computadora, sino también por agregar retrasos en la transmisión de las señales. Todos estos sistemas manejan un amplio conjunto de dispositivos compatibles que pueden ser buenos a primera vista, pero hacen que algunas características específicas del hardware no estén disponibles por razones de compatibilidad. Del lado del hardware de código abierto, podemos encontrar que OpenBCI es una opción flexible, pero con algunas carencias importantes. La más importante se basa en que la comunicación entre la computadora y la placa no siempre es estable y su \gls*{GUI} no brinda la posibilidad de adquirir datos bajo un paradigma \gls*{BCI} particular. Por otro lado, su base de hardware y las características de \gls*{SDK} le dan a esta placa un gran potencial para implementar un sistema \gls*{BCI} completo comparable con el equipo de grado médico.\\


% With all these factors in mind, we aim to develop a standalone \gls*{BCI} system with the OpenBCI Cyton board that handles the signal acquisition and the stimuli deliver in the same interface, to reduce the needed infrastructure to perform neurophysiological experiments. Alongside a distributed platform to improve the performance, increase the scalability, and reduce the jitter. This software, BCI-Framework, provides the user with a built-in development environment enhanced with a custom API for data interactions, montage context, and markers generation. This environment is full compatible with any Python module and is focused in the generation of real-time visualizations, data analysis and network-based stimuli delivery for the remote presentation of audiovisual cues. This approach converges almost all needed components for \gls*{BCI} researches into a single standalone implementation.\\

Con todos estos factores en mente, nuestro objetivo es desarrollar un sistema \gls*{BCI} independiente con la placa OpenBCI Cyton que maneje la adquisición de señales y la entrega de estímulos en la misma interfaz, para reducir la infraestructura necesaria para realizar experimentos neurofisiológicos. Junto con una plataforma distribuida para mejorar el rendimiento, aumentar la escalabilidad y reducir el \textit{jitter}. Este software, BCI-Framework, proporciona al usuario un entorno de desarrollo integrado mejorado con una API personalizada para interacciones de datos, selección de montaje y generación de marcadores. Este entorno es totalmente compatible con cualquier módulo de Python y se centra en la generación de visualizaciones en tiempo real, análisis de datos y entrega de estímulos a través de conexiones de red para la presentación remota de señales audiovisuales. Este enfoque reúne casi todos los componentes necesarios para la investigación de \gls*{BCI}

% In a nutshell, the introduced EEG-based \gls*{BCI} framework comprises the following benefits: i) A portable and cheap acquisition system (hardware) founded on the well-known Open-BCI devices. ii) This approach includes a wireless, e.g., Wi-Fi, communication protocol to couple the EEG data acquisition and event markers synchronization from audiovisual stimulation paradigms. iii) A distributed system is enhanced within this \gls*{BCI} framework to carry out real-time data acquisition and visualization while favoring the inclusion of conventional or user-designed EEG data processing libraries over a Python language environment. This \gls*{BCI} framework fundamentals are listed below regarding the hardware and software design main insights. Besides, a latency-based quality assessment is carried for method comparison.\\

En pocas palabras, el sistema \gls*{BCI} basado en EEG presentado comprende los siguientes beneficios: i) Un sistema de adquisición portátil y económico (hardware) basado en los conocidos dispositivos OpenBCI. ii) Este enfoque incluye un protocolo de comunicación inalámbrico (Wi-Fi), para acoplar la adquisición de datos de EEG y la sincronización de marcadores de eventos de paradigmas de estimulación audiovisual. iii) Se implementa un sistema distribuido dentro de este entorno \gls*{BCI} para llevar a cabo la adquisición y visualización de datos en tiempo real mientras se favorece la inclusión de bibliotecas de procesamiento de datos EEG convencionales o diseñadas por el usuario sobre un entorno de lenguaje Python. Además, se lleva a cabo método para la evaluación de la calidad basada en la latencia.\\

\\[2.0cm]

\textbf{\small Palabras clave:} Interfaces Cerebro-Computador, Adquisición de señales, Experimentos neurofisiológicos, Sistemas distribuidos, Sistemas embebidos, OpenBCI.\\