\cleardoublepage
\chapter*{Abstract}
\markboth{Abstract}{}
\addcontentsline{toc}{chapter}{Abstract}

The widespread use of neurophysiological signals to develop \gls*{BCI} systems has certainly varied clinical and nonclinical applications. Main implementations in medical issues include: rehabilitation, cognitive state analysis, diagnostics, assistive devices for communication, locomotion and movement. By other hand, there is a bunch of researches that approaches the \gls*{BCI} systems to healthy people in fields like: neuroergonomics, smart homes, neuromarketing and advertising, games, education, entertainment and even security and validation. Not all EEG acquisition systems are capable to use in \gls*{BCI}s systems. Even if the clinic devices are highly accurate, these implementations have a limited, or nonexistent, real-time data flow access; because they mainly use is about diagnostic and offline analysis. Recently, and because of the cheapening prototyping development, there is in the market a set of low-cost embedded systems for \gls*{EEG} acquisition, i.e., OpenBCI, InteraXon, Muse, NeuroSky MindWave and Emotiv. All these options usually include a high or low-level \gls*{SDK}, that could be open-source or proprietary and will come with a different grade of flexibility (rigid or customizable electrode placement, multiple sampling rates, transmission protocols, wireless, etc). Many of these devices have shown capabilities to handle \gls*{BCI} tasks, but they need a context-specific development to boost their base benefits. Acquiring brain signals is only one task for a \gls*{BCI} system, also it is necessary to carry out a lot of data processing and controlled experiments, concerning this have been specialized software for developers and researchers purpose i.e., BCI2000, Neurobehavioral Systems Presentation, Psychology Software Tools, Inc. ePrime and PsychoPy. All these systems offer greater ease of use through experimenter interfaces, but they can be costly, require high-level programming and technical skills, and usually do not support dedicated data acquisition. For this reason, the acquisition involves the implementation of third party software and drivers; consequently, losing interesting hardware features in favor to support as many devices as possible. To implement a \gls*{BCI} system is an interdisciplinary activity that requires a set of specific and outstanding knowledges about communication systems, signals acquisition, instrumentation, clinical protocols, experiments validation, software development, among others.\\

Besides, in order to perform a real-world experiment, the user must calibrate the specific set of acquisition system, stimuli delivery and data processing stages. Current software approaches try to converge multiple technologies and methodologies to provide general purpose \gls*{BCI} systems. The most popular is the BCI200, which comes with default paradigms but their interface has been pointed out to be not very intuitive and its operation is difficult to understand, although, it is possible to add new paradigms, this include software contributions using their own libraries and do not through a built-int development interface. Other software widely used is the OpenVIBE this one includes a graphical drag-and-drop interface to perform data analysis with an extensive set of pre-defined algorithms. Its synchronous acquisition system is known for not only occasionally frozen the computer but also for adding delays to the streaming of the signals. All these systems handle with an extensive set of compatible devices which may be good at first glance but make that some specific hardware features are not available for compatibility reasons. On the side of the open source hardware, we can find that OpenBCI a flexible option, but with some important lacks. The most important relies on the communication between the computer and the board is not always stable and their \gls*{GUI} does not provide the possibility of acquiring data under wich a particular \gls*{BCI} paradigm. Otherwise, their hardware base and \gls*{SDK} features gives to this board a huge potential to implement a complete \gls*{BCI} system comparable with medical grade equipment.\\

With all these factors in mind, we aim to develop a standalone \gls*{BCI} system with the OpenBCI Cyton board that handles the signal acquisition and the stimuli deliver in the same interface, to reduce the needed infrastructure to perform neurophysiological experiments. Alongside a distributed platform to improve the performance, increase the scalability, and reduce the \textit{jitter}. This software, BCI-Framework, provides the user with a built-in development environment enhanced with a custom API for data interactions, montage context, and markers generation. This environment is full compatible with any Python module and is focused in the generation of real-time visualizations, data analysis and network-based stimuli delivery for the remote presentation of audiovisual cues. This approach converges almost all needed components for \gls*{BCI} researches into a single standalone implementation.\\

In a nutshell, the introduced EEG-based \gls*{BCI} framework comprises the following benefits: i) A portable and cheap acquisition system (hardware) founded on the well-known OpenBCI devices. ii) This approach includes a wireless, e.g., Wi-Fi, communication protocol to couple the EEG data acquisition and event markers synchronization from audiovisual stimulation paradigms. iii) A distributed system is enhanced within this \gls*{BCI} framework to carry out real-time data acquisition and visualization while favoring the inclusion of conventional or user-designed EEG data processing libraries over a Python language environment. In addition, a latency-based quality assessment method is carried out.

\\[2.0cm]

\textbf{\small Keywords:} Brain-Computer Interface, Signals acquisition, Neurophysiological experiments, Distributed systems, Embedded systems, OpenBCI.\\
